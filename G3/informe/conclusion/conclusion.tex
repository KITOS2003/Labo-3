

%que se hizo?
%resultadosS$ 
%es correcto si no por que
%--------------serie------------------
\paragraph{}En el circuito RLC en serie las frecuencias de resonancia fueron muy similares siendo $f_{0_1}= (1590.22 \pm 0.03)\,$Hz, $f_{0_2}=(1590.08 \pm 0.09)\,$Hz y $f_{0_3}=(1597.62 \pm 0.61)\,$Hz. 
También se observó una disminución en $Q$ siendo $Q_1= 20.49 \pm 0.02$, $Q_2 = (6.84 \pm 0.01)$ y $Q_3 = 0.97 \pm 0.01$. Como puede verse $Q_1 > Q_2 > Q_3$, lo cual es esperable que ocurra con un aumento de la resistencia del sistema. El modelo no se ajustó correctamente a los datos en frecuencias altas.
%--------- paralelo ------------------
\paragraph{}Para el RLC en paralelo, la frecuencia $f_{0||}$ para el primer valor de resistencia fue de $f_{0_1} = (1589.41 \pm 0.03)\,$Hz, y para el segundo, $f_{0_2} = (1584.22 \pm 0.05)\,$Hz. Los factores de mérito fueron $Q_1=31.22 \pm 0.03$ y $Q_2=15.12 \pm 0.03$ respectivamente, y estos cumplen con la relación dada por los valores de las resistencias y la expresión \eqref{eq:Q RLC}. El modelo no se ajustó correctamente a los datos en las frecuencias altas.
%---------final thoughts--------------
\paragraph{}
El modelo de parámetros concentrados demostró ser una buena aproximación para describir circuitos RLC serie y RLC paralelos alimentados con fuentes de tensión armónica con frecuencias por debajo de los $10\,\mathrm{kHz}$, de dicha frecuencia en adelante se observan diferencias significativas con respecto al modelo cada vez mayores a medida la frecuencia aumenta.
Además se observó que en las frecuencias altas ambos circuitos presentan una caída de su atenuación con respecto de el valor predicho por el modelo, por lo que éste no es un modelo adecuado para describir el sistema a partir del orden de $10^5\,$Hz.