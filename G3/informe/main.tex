\documentclass{article}
\usepackage[utf8]{inputenc}
\usepackage{graphicx}
\usepackage{float}
% \graphicspath{ {./images/} }
\usepackage[spanish]{babel}
\usepackage[dvipsnames]{xcolor}
\usepackage{biblatex}
\usepackage[bookmarks = true, colorlinks=true, linkcolor = black, citecolor = black, menucolor = black, urlcolor = black]{hyperref} 
\usepackage{amsmath}
\usepackage{multicol}
\usepackage{geometry}
\usepackage{csquotes}
\usepackage{subfiles}
\usepackage{siunitx}

%\usepackage[bookmarks = true, colorlinks=true, linkcolor = black, citecolor = black, menucolor = black, urlcolor = black]{hyperref} <---este choclazo te saca el recuadro rojo de los links
\addbibresource{bibliografia.bib}

\geometry{
a4paper,
total = {170mm,257mm},
left = 22mm,
right = 22mm,
top = 20mm
}

\title{Estudio de la resonancia y antiresonancia de circuitos RLC}
\author{Marcos Sidoruk \\ \href{mailto:marcsid2003@gmail.com}{marcsid2003@gmail.com} \and Gaspar Casaburi \\ \href{mailto:gaspar.casaburi@gmail.com}{gaspar.casaburi@gmail.com} \and Candelaria Rico \\ \href{mailto:canderico78@gmail.com}{canderico78@gmail.com}}
\date{Febrero 2023}

\begin{document}

\maketitle

\begin{abstract}
    \subfile{Resumen/resumen.tex}
\end{abstract}

\section{Introducción}

\subfile{introduccion/intro.tex}

\section{Desarrollo experimental y resultados}
\paragraph{}
En este trabajo se estudió la resonancia de un circuito RLC en serie (\ref{sec:RLC serie}) y la antirresonancia de un circuito RLC en paralelo (\ref{sec:RLC paralelo}). Además, se graficaron los diagramas de Bode de ambos circuitos.
\subfile{desarrollo/exp1.tex}
\subfile{desarrollo/exp2.tex}


\section{Conclusiones}

\subfile{conclusion/conclusion.tex}

%$$ A = \frac{R_0V_0}{RQ} $$

%$$ V_{ef}(\omega) = \frac{R_{lim}V_0}{|R_{lim} + Z|} $$

%$$ V_{ef}(\omega) = \frac{A}{\sqrt{\big(
%B + \frac{(\frac{\omega}{\omega_0})^2}{\frac{1}{Q^2}+(\frac{\omega}{\omega_0} - \frac{\omega_0}{\omega})^2}
%\big)^2 + \frac{1}{Q^2}\big(
%\frac{Q^2(\frac{\omega}{\omega_0} - \frac{\omega_0}{\omega}) +  \frac{\omega_0}{\omega}}{\frac{1}{Q^2}+(\frac{\omega}{\omega_0} %- \frac{\omega_0}{\omega})^2}
%\big)^2}} $$

%$$ A = \frac{R_{lim}V_0}{R} $$
%$$ B = \frac{R_{lim}}{R} $$

%$$ \Delta\phi(\omega) = \arctan\bigg(\frac{\Im\frac{V_{medido}}{V_{emitido}}}{\Re\frac{V_{medido}}{V_{emitido}}}\bigg) $$

%$$ \Delta\phi(\omega) = \arctan\bigg(\frac{1}{Q}\frac{
%Q^2(\frac{\omega}{\omega_0} - \frac{\omega_0}{\omega}) +  \frac{\omega_0}{\omega}
%}{
%B(\frac{1}{Q^2} + (\frac{\omega}{\omega_0} - \frac{\omega_0}{\omega})^2) + (\frac{\omega}{\omega_0})^2
%}\bigg)$$

\printbibliography

\end{document}
