\documentclass{article}
\usepackage[utf8]{inputenc}
\usepackage{graphicx}
\usepackage{float}
\usepackage[spanish]{babel}
\usepackage[dvipsnames]{xcolor}
\usepackage{biblatex}
\usepackage[bookmarks = true, colorlinks=true, linkcolor = black, citecolor = black, menucolor = black, urlcolor = black]{hyperref} 
\usepackage{amsmath}
\usepackage{multicol}
\usepackage{geometry}
\usepackage{csquotes}
\usepackage{subfiles}
\usepackage[separate-uncertainty=true]{siunitx}

\addbibresource{bibliografia.bib}

\geometry{
a4paper,
total = {170mm,257mm},
left = 22mm,
right = 22mm,
top = 20mm
}

\title{Caracterización del régimen transitorio en circuitos RC, RL y RLC}
\author{Marcos Sidoruk \\ \href{mailto:marcsid2003@gmail.com}{marcsid2003@gmail.com} \and Gaspar Casaburi \\ \href{mailto:gaspar.casaburi@gmail.com}{gaspar.casaburi@gmail.com} \and Candelaria Rico \\ \href{mailto:canderico78@gmail.com}{canderico78@gmail.com}}
\date{Febrero 2023}

\begin{document}

\maketitle

\begin{abstract}
    \subfile{0.Abstract/Abstract.tex}
\end{abstract}

\section{Introducción}
\subfile{1.Introduccion/Introduccion.tex}

\section{Desarrollo experimental y Resultados}
En este trabajo se estudiaron los regímenes transitorios de circuitos RC (sección \ref{sec: RC}), RL (sección \ref{sec: RL}) y RLC (sección \ref{sec: RLC}). Se utilizaron un generador de funciones Tektronix AFG1022 \cite{manual_generador} y un osciloscopio Tektronix TBS 1052B-EDU \cite{manual_osciloscopio}. Todas las resistencias fueron medidas con un óhmetro modelo UNI-T UT55\cite{manual_multimetro}, y las capacitancias e inductancias fueron medidas con un multímetro Extech LCR Meter \cite{manual_extech}.
\subfile{2.Desarrollo/RC.tex}
\subfile{2.Desarrollo/RL.tex}
\subfile{2.Desarrollo/RLC.tex}
\section{Conclusión}
\subfile{3.Conclusiones/Conclusiones.tex}

\printbibliography

\end{document}
