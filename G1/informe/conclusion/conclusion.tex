\paragraph{}
En este trabajo se estudiaron y determinaron las resistencias internas del voltímetro y amperímetro de un multímetro comercial, la resistencia del voltímetro de un osciloscopio y la de una fuente. Los resultados obtenidos fueron consistentes con los valores oficiales publicados por el fabricante, en el caso de que fueran reportados, y coherentes con la funcionalidad de cada elemento estudiado.

\paragraph{}
Las resistencias internas de ambos voltímetros se encontraron del orden de los M$\Omega$. En el voltímetro del multímetro (sección \ref{sec:resistencia de volt}), se realizó un ajuste utilizando la ecuación \ref{eqn:caida voltimetro} y se encontró un valor para la resistencia de $R_v=(10.01\pm0.02)$ M$\Omega$, que coincide con lo reportado por el fabricante. En el voltímetro integrado en el osciloscopio (sección \ref{sec:resistencia de osc}), se obtuvo un valor de $R_{osc}=(1.009 \pm 0.002)$ M$\Omega$. Si bien no se encontró el valor publicado por el fabricante, se puede afirmar que el resultado es coherente con el esperado para cualquier voltímetro, dado que se trata de un instrumento en el que se espera medir diferencias de potencial, y esto requiere que pase la menor cantidad de corriente posible. 

\paragraph{}
En la sección \ref{sec:resistencia de amp}, se calculó la resistencia interna del amperímetro. Se encontró una resistencia interna de $R_a=(1.13 \pm 0.01)$ $\Omega$. Este valor es esperable dado que en este instrumento se espera medir la corriente que pasa por un cable sin modificarla. Cabe mencionar que, para que el amperímetro funcione de manera adecuada, la resistencia total del circuito debe ser mucho mayor a su resistencia interna.

\paragraph{}
En el caso de la fuente (sección \ref{sec:resistencia de fuente}), se encontró una resistencia interna de $R_i=(51,4 \pm 0,5)$ $\Omega$. Nuevamente, este resultado es adecuado para una fuente comercial, dado que requiere una resistencia interna de esta magnitud para no afectar el resto de los elementos del circuito.

%Las resistencias calculadas son las reportadas por el fabricante y tienen logica 

%que se hizo?
%resultados
%es correcto si no por que
