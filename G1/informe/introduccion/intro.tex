\paragraph{}
Un circuito eléctrico contiene componentes que transportan corriente. Estos elementos afectan a la totalidad del circuito de diferentes maneras. Las resistencias, por ejemplo, generan una diferencia de potencial entre sus extremos. Todo circuito sigue las leyes de Kirchhoff enunciadas a continuación. 

\paragraph{}
La primera ley de Kirchhoff o \emph{ley de nodos} establece que la suma de las corrientes que entran o salen de un nodo es nula. Esta ley se describe con la ecuación \ref{eqn:ley de nodos},\cite{roederer}
\begin{equation}\label{eqn:ley de nodos}
    \sum_{k} I_k=0
\end{equation}
donde $I_k$ es cada corriente que entra o sale del nodo, con su correspondiente signo.
\paragraph{}
La segunda ley de Kirchhoff o \textit{ley de mallas} establece que la suma algebraica de las diferencias de potencial eléctrico en un circuito cerrado es igual a cero.

\paragraph{}
Una resistencia es un componente eléctrico dentro de un circuito que se opone al flujo de corriente. 
Si en un circuito hay múltiples resistencias en serie, se puede calcular una resistencia equivalente como una suma algebraica de las mismas. \cite{roederer}
\begin{equation}\label{eqn: en serie}
    \sum_k R_k=R_{eq}
\end{equation}
Si, en cambio, las resistencias se encuentran en paralelo, se puede calcular una resistencia equivalente como en la ecuación \ref{eqn: en paralelo}.
\begin{equation}\label{eqn: en paralelo}
    \sum_k R_{k}^{-1}=R_{eq}^{-1}
\end{equation}

\paragraph{}
Bajo ciertos parámetros, las resistencias de un circuito satisfacen la ley de Ohm, ya que en éstas se cumple que la caída de potencial es proporcional a la corriente que circula por ellas. Esta ley esta descripta en la ecuación \ref{eqn:ley de ohm},

\begin{equation}\label{eqn:ley de ohm}
    \Delta V=I R
\end{equation}
siendo $\Delta V$ la diferencia de potencial, $I$ la corriente y $R$ la resistencia. 

\paragraph{}
La potencia disipada en una resistencia puede calcularse mediante la siguiente fórmula
\begin{equation}\label{potencia}
    P = V I
\end{equation} 
donde $V$ es la diferencia de potencial e $I$ la corriente. Utilizando la ley de Ohm (\ref{eqn:ley de ohm}), podemos reescribir esta ecuación en términos de la diferencia de potencial y la resistencia, como muestra la ecuación \ref{eq_fuente_1}.

\begin{equation}\label{eq_fuente_1}
    P = \frac{V^2}{R_C}
\end{equation}


\paragraph{}
Existen diferentes instrumentos cuya función en medir magnitudes de un circuito. Su funcionamiento se basa en la ley de Ohm. Estos instrumentos cuentan con una resistencia interna acorde con su fin.

\paragraph{}
Un voltímetro es un instrumento de medición que permite medir la diferencia de potencial electrostático entre dos puntos de un circuito. Para ello, se debe conectar dicho instrumento en paralelo a los dos puntos en los cuales se desea medir. Para que dicha conexión interfiera lo menos posible con las variables dinámicas del circuito, el voltímetro debe de estar dotado con una resistencia interna \textit{mucho mayor} respecto a las del circuito a estudiar.

\paragraph{}
Un amperímetro es un instrumento cuyo propósito es medir la corriente que circula a través de él. A diferencia del voltímetro, éste debe conectarse en serie con los demás elementos del circuito y su resistencia interna debe ser \textit{despreciable} con respecto a las resistencias del circuito a estudiar.

\paragraph{} La fuente es un elemento activo capaz de generar una diferencia de potencial entre sus extremos. Debe contener una resistencia interna despreciable respecto a las resistencias del circuito a estudiar.


\paragraph{}
En este trabajo se determinaron las resistencias internas de un voltímetro, un amperímetro, un osciloscopio y la fuente utilizada.

%\paragraph{}
%Al mismo tiempo, la segunda ley de Kirchoff exige que la corriente sea:
%$$I = \frac{E}{r_i + R_C}$$
%Con lo que también podemos escribir la potencia como:



